%%%%%%%%%%%%%%%%%%%%%%%%%%%%%%%%%%%%%%%%%
% Twenty Seconds Resume/CV
% LaTeX Template
% Version 1.0 (14/7/16)
%
% This template has been downloaded from:
% http://www.LaTeXTemplates.com
%
% Original author:
% Carmine Spagnuolo (cspagnuolo@unisa.it) with major modifications by
% Vel (vel@LaTeXTemplates.com)
%
% License:
% The MIT License (see included LICENSE file)
%
%%%%%%%%%%%%%%%%%%%%%%%%%%%%%%%%%%%%%%%%%

%----------------------------------------------------------------------------------------
%	PACKAGES AND OTHER DOCUMENT CONFIGURATIONS
%----------------------------------------------------------------------------------------

\documentclass[letterpaper]{twentysecondcv} % a4paper for A4

% Command for printing skill progress bars
\newcommand\skills{
~
	\smartdiagram[bubble diagram]{
        \textbf{Web}\\\textbf{Development},
        \textbf{Machine}\\\textbf{~~~~~Learning~~~~~},
        \textbf{~~~~~OOP~~~~~~},
        \textbf{Deep}\\\textbf{~~~~Learning~~~~},
        \textbf{Open Source}\\\textbf{Development},
        \textbf{~Full Stack~},
        \textbf{Services}\\\textbf{~~and APIs~~}
    }
}

\programming{{C $\textbullet$ C++ $\textbullet$ \large \LaTeX / 3}, {R $\textbullet$ Rails $\textbullet$ Shell $\textbullet$ Java / 4}, {C\# $\textbullet$ Python $\textbullet$ PyTorch / 5}, {HTML5 $\textbullet$ TypeScript $\textbullet$ JS / 5.5}}

\newcommand\projects{
    % Custom \profilesection
    {\noindent\color{black!80} \LARGE Projects \& Conferences \rule[0.15\baselineskip]{2cm}{1pt} \vspace{-5pt}}

    {\begin{itemize}  \itemsep -18pt %reduce space between items
      \item {[YouTube]} \href{https://youtu.be/-6Aq5Q0c-CU}{@deephypebot final project presentation at OpenAI} \\
      \item {[GitHub]} \href{https://github.com/iconix/fast.ai}{fast.ai coursework \& notes} \\
      \item {[Blog]} \href{https://iconix.github.io/portfolio-building/2017/09/25/nlp-for-tasks.html}{An ML model for task extraction} \\
      \item {[Blog]} \href{https://iconix.github.io/portfolio-building/2017/10/18/data-is-all-around-us.html}{A ported Python analysis of \textit{Love Actually}} \\
      \item {[Blog]} \href{https://iconix.github.io/notes/2017/12/07/topics-and-dim-reduction.html}{A guide to topic models \& dimensionality reduction} \\
      \item \href{http://www.widsconference.org/}{\textit{Women in Data Science Conference}} - Feb 3, 2017 \& Mar 5, 2018 (online) \\
      \item \href{https://ghc.anitab.org/}{\textit{Grace Hopper Celebration of Women in Computing}} - Oct 19-21, 2016 \\
      \item \href{http://www.femalefoundersconference.org/}{\textit{Y Combinator's Female Founders Conference}} - Mar 1, 2014
    \end{itemize}}
}

%----------------------------------------------------------------------------------------
%	 PERSONAL INFORMATION
%----------------------------------------------------------------------------------------

% If you don't need one or more of the below, just remove the content leaving the command, e.g. \cvnumberphone{}

\cvname{Nadja Rhodes} % Your name
\cvjobtitle{ Software/ML Engineer } % Job
% title/career

\cvlinkedin{/in/nadjarhodes}
\cvgithub{iconix}
\cvsite{iconix.github.io} % Personal website
\cvnumberphone{} % Phone number
\cvmail{narhodes1+res@gmail.com} % Email address

%----------------------------------------------------------------------------------------

\begin{document}

\makeprofile % Print the sidebar

%----------------------------------------------------------------------------------------
%	 EDUCATION
%----------------------------------------------------------------------------------------
\section{Education}

\begin{twenty} % Environment for a list with descriptions
	\twentyitem
    	{2009 - 2013}
		{}
        {B.S. Computer Science}
        {Stanford, CA}
        {}
        {Stanford University, School of Engineering}
	%\twentyitem{<dates>}{<title>}{<organization>}{<location>}{<description>}
\end{twenty}

%----------------------------------------------------------------------------------------
%	 EXPERIENCE
%----------------------------------------------------------------------------------------

\section{Experience}

\begin{twenty} % Environment for a list with descriptions
    \twentyitem
    	{Jun 2018 -}
		{Sep 2018}
        {Deep Learning Scholar}
        {\href{https://blog.openai.com/openai-scholars/}{OpenAI}}
        {}
        {
        {\begin{itemize} \itemsep -2pt %reduce space between items
        \item Granted a scholarship to study deep learning full time under mentorship by \href{https://www.linkedin.com/in/natashajaques}{Natasha Jaques} (MIT, DeepMind). Focused on techniques in language modeling and generative text for natural language processing (NLP).
        \item Developed and \href{https://github.com/iconix/deephypebot}{open-sourced} a deep generative language model trained on past human music writing, compiled from the web and conditioned on attributes of the referenced music. Model outputs are deployed live via \href{https://twitter.com/deephypebot}{@deephypebot}, a Twitter bot that automatically finds ongoing discussions about music and generates new music commentary.
        \end{itemize}}
        }
    \\
    \twentyitem
    	{Sep 2013 -}
		{Sep 2018}
        {Software Engineer 2}
        {Microsoft, \href{http://www.onenote.com/}{OneNote Services}}
        {}
        {
        {\begin{itemize} \itemsep -2pt %reduce space between items
        \item Started a machine learning and NLP incubation effort with three other engineers. Delivered prototypes in the note-taking domain.
        \item Rebuilt and open-sourced the \href{https://github.com/OneNoteDev/WebClipper}{OneNote Web Clipper} (a multi-browser extension for extracting web content into OneNote) on a modern, React-like library. Primary architect of V1 data collection infrastructure, enabling real-time health monitoring and data-driven decision making about where to invest next in the product.
        \item Primary developer for V1 of \href{https://www.onenote.com/notebooks}{onenote.com/notebooks}, the main entry point for 4M monthly active users to find and access their OneNote content online.
        \end{itemize}}
        }

	%\twentyitem{<dates>}{<title>}{<location>}{<description>}
\end{twenty}

%----------------------------------------------------------------------------------------
%	 OTHER
%----------------------------------------------------------------------------------------

\section{Self-Study}
\begin{itemize} \itemsep -2pt %reduce space between items
        \item Practical Deep Learning for Coders by Jeremy Howard (fast.ai) - {\footnotesize{Completed 2018}}
        \item Machine Learning by Stanford University/Andrew Ng (Coursera) - {\footnotesize{Completed 2014}}
       	\item R Programming by Johns Hopkins University (Coursera) - {\footnotesize{Completed 2016}}
        \item The Data Scientist's Toolbox by Johns Hopkins University (Coursera) - {\footnotesize{Completed 2016}}
\end{itemize}


\section{Internships}
\begin{twenty}
	\twentyitem
    	{Summer}
        {2012}
        {Trainee}
        {\href{http://www.kecl.ntt.co.jp/rps/index.html}{NTT Communication Science Laboratories}, Japan}
        {}
        {\begin{itemize} \itemsep -2pt %reduce space between items
        \item Linguistic Intelligence Research Group
       	\item Implemented additional features to standard sequential pattern mining algorithm (PrefixSpan), introduced to distributed computing via Hadoop.
        \end{itemize}}
    \\
    \twentyitem
    	{Summer}
        {2011}
        {Intern (BOLD Engineering Practicum)}
        {Google, Inc.}
        {}
        {\begin{itemize}  \itemsep -2pt %reduce space between items
        \item Google AdWords Display Network
        \item Developed and released client-motivated UI feature for Contextual Targeting Tool,
            used to build advertising campaigns.
        \end{itemize}}
    \\
    \twentyitem
    	{Summer}
        {2010}
        {Research Assistant, Sociology Department}
        {Stanford University}
        {}
        {\begin{itemize}  \itemsep -2pt %reduce space between items
        \item Supervised by \href{https://www.chicagobooth.edu/faculty/directory/s/amanda-j-sharkey}{Dr. Amanda J. Sharkey}
        \item Optimized manual data mining project by developing Python code to automate
            LexisNexis Academic news database searches.
        \end{itemize} }
\end{twenty}

\end{document}
